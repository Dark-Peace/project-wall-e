\documentclass[utf8]{article}

\usepackage[utf8]{inputenc}

\usepackage[parfill]{parskip}

\usepackage{amsmath}
\usepackage{mathtools}
\usepackage{nccmath}
\usepackage{amssymb}
\usepackage{amsfonts}
\usepackage{graphicx}
\usepackage{caption}
\usepackage{subcaption}
\usepackage{float}
\usepackage{listingsutf8}
\usepackage{hyperref}
\usepackage[dvipsnames]{xcolor}
\usepackage{comment}

\usepackage{fullpage}

%------------------------------------------------------

\usepackage{listings}
\usepackage{adjustbox}

\definecolor{color0}{RGB}{147, 147, 147}
\definecolor{color1}{RGB}{186, 033, 033}
\definecolor{color2}{RGB}{000, 128, 000}
\definecolor{color3}{RGB}{064, 128, 128}
\definecolor{color4}{RGB}{170, 034, 255}

\lstdefinelanguage{clips}{
  mathescape = true,
  sensitive        = true,
  morecomment      = [l]{;},
  showstringspaces = false,
  morestring       = [b]",
}

% egreg's modulo macro (see https://tex.stackexchange.com/a/34449/21891)
\def\truncdiv#1#2{((#1-(#2-1)/2)/#2)}
\def\moduloop#1#2{(#1-\truncdiv{#1}{#2}*#2)}
\def\modulo#1#2{\number\numexpr\moduloop{#1}{#2}\relax}


\makeatletter

% a TeX counter to keep track of the nesting level
\newcount\netParensCount@clisp

% Modify how ( and ) get typeset depending on the value of the counter
% (Based on Ulrike Fischer's approach to modifying characters in listings;
% see https://tex.stackexchange.com/a/231927/21891)
\lst@CCPutMacro
\lst@ProcessOther{`(}{{%
  \ifnum\lst@mode=\lst@Pmode\relax%
    \rainbow@clisp{(}%
    \global\advance\netParensCount@clisp by \@ne%
  \else
    (%
  \fi
}}%
\lst@ProcessOther{`)}{{%
  \ifnum\lst@mode=\lst@Pmode\relax%
    \global\advance\netParensCount@clisp by \m@ne%
    \rainbow@clisp{)}%
  \else
    )%
  \fi
}}%
\@empty\z@\@empty

% Color its argument based on the value of the \netParensCount@clisp counter
% (modulo 5)
\newcommand\rainbow@clisp[1]{%
  \ifcase\modulo\netParensCount@clisp 5\relax%
    \textcolor{color0}{#1}%
  \or
    \textcolor{color1}{#1}%
  \or
    \textcolor{color2}{#1}%
  \or
    \textcolor{color3}{#1}%
  \else
    \textcolor{color4}{#1}%
  \fi
}

\lst@AddToHook{PreInit}{%
  \global\netParensCount@clisp 0\relax%
}

\makeatother

\lstnewenvironment{clips-code}
  {\lstset{language=clips}}
  {}

\setcounter{tocdepth}{6}
\setcounter{secnumdepth}{6}

% -----------------------------------------------------


\title{INFO-F-305 - Modélisation et Simulation
\\Projet Octave
\\Wall-e}
\author{Becker Robin Gilles - Bourgeois Noé}
\date{December 2022}

\begin{document}
\maketitle
\tableofcontents

\newpage

% -----------------------------------------------------

\section{Introduction}
\newpage

\section{Modélisation}
\newpage

\section{Analyse de cas}
\newpage

\subsection{Les sentiments d’EVE ne dépendent pas de WALL-E}

\begin{equation}
\centering
\left\{\begin{split}
\dot{w}(t) &= aw(t) + be(t)\\
\dot{e}(t) &= 0 \\
\end{split}\right.
 \end{equation}

\begin{equation}
\centering
A = \left[
\begin{array}{cc}
a & b\\
0 & 0
\end{array}
\right]
 \end{equation}

\begin{figure}
\centering
\begin{subfigure}{.5\textwidth}
  \centering
  \includegraphics[width=\linewidth]{pics/11}
  \caption{A subfigure}
  \label{fig:sub1}
\end{subfigure}%
\begin{subfigure}{.5\textwidth}
  \centering
  \includegraphics[width=\linewidth]{pics/12}
  \caption{A subfigure}
  \label{fig:sub2}
\end{subfigure}
\caption{A figure with two subfigures}
\label{fig:test}
\end{figure}

\begin{figure}
\centering
  \begin{subfigure}{.5\textwidth}
  \centering
  \includegraphics[width=\linewidth]{pics/11_-2_2}
  \caption{A subfigure}
%  \label{fig:sub1}
\end{subfigure}%
\begin{subfigure}{.5\textwidth}
  \centering
  \includegraphics[width=\linewidth]{pics/11_-2_2}
  \caption{A subfigure}
%  \label{fig:sub2}
\end{subfigure}
\caption{A figure with two subfigures}
%\label{fig:test}
\end{figure}

\newpage

\subsection{Les deux robots ont la même dynamique}

\begin{equation}
\centering
\left\{\begin{split}
\dot{w}(t) &= aw(t) + be(t)\\
\dot{e}(t) &= bw(t) + ae(t) \\
\end{split}\right.
 \end{equation}

On peut en déduire la matrice à coefficients A telle que X˙ (t) = Ax :

% coefficients matrix A
\begin{equation}
\centering
A = \left[
\begin{array}{cc}
a & b\\
b & a
\end{array}
\right]
 \end{equation}

De là, nous pouvons décider d’utliser le diagramme de Pointcarré
ou la méthode des valeurs propres
pour définir le type de système (selle, nœud stable, ….)
Dans cet exemple, nous allons utiliser
les valeurs propres.
On calcule les valeurs propres en posant l’équation caractéristique :

\begin{equation}
\centering
\det(A − λI) = 0
 \end{equation}

On résoud et on obtient les valeurs propres :

\begin{equation}
\centering
\lambda_{1,2} = 2a \pm \sqrt{4b^2} = {a - b, a + b}
 \end{equation}

On distingue les différents cas possibles :
1. Selle : pour que le système soit une selle, il faut

\begin{equation}
\centering
\lambda_{1,2} \neq 0
 \end{equation}

et
\begin{equation}
\centering
signe \lambda_{1} \neq signe \lambda_{2}
 \end{equation}

On a donc
a < b
et -b < a
ou
b < a
et a < -b


\begin{figure}
\centering
\begin{subfigure}{.5\textwidth}
  \centering
  \includegraphics[width=\linewidth]{pics/18}
  \caption{a < b
et -b < a}
%  \label{fig:sub1}
\end{subfigure}%
\begin{subfigure}{.5\textwidth}
  \centering
  \includegraphics[width=\linewidth]{pics/18-2}
  \caption{b < a
et a < -b }
%  \label{fig:sub2}
\end{subfigure}
\caption{A figure with two subfigures}
%\label{fig:test}
\end{figure}

\begin{figure}
\centering
  \begin{subfigure}{.5\textwidth}
  \centering
  \includegraphics[width=\linewidth]{pics/21_-2_2}
  \caption{A subfigure}
%  \label{fig:sub1}
\end{subfigure}%
\begin{subfigure}{.5\textwidth}
  \centering
  \includegraphics[width=\linewidth]{pics/21_-2_2}
  \caption{A subfigure}
%  \label{fig:sub2}
\end{subfigure}
\caption{A figure with two subfigures}
%\label{fig:test}
\end{figure}

\newpage

\subsection{Les deux robots auront une dynamique contrastante}


\begin{equation}
\centering
\left\{\begin{split}
\dot{w}(t) &= aw(t) + be(t)\\
\dot{e}(t) &= −bw(t) − ae(t) \\
\end{split}\right.
 \end{equation}

\begin{equation}
\centering
A = \left[
\begin{array}{cc}
a & b\\
-b & -a
\end{array}
\right]
 \end{equation}

\begin{figure}
\centering
\begin{subfigure}{.5\textwidth}
  \centering
  \includegraphics[width=\linewidth]{pics/31}
  \caption{A subfigure}
%  \label{fig:sub1}
\end{subfigure}%
\begin{subfigure}{.5\textwidth}
  \centering
  \includegraphics[width=\linewidth]{pics/32}
  \caption{A subfigure}
%  \label{fig:sub2}
\end{subfigure}
\caption{A figure with two subfigures}
%\label{fig:test}
\end{figure}

\begin{figure}
\centering
  \begin{subfigure}{.5\textwidth}
  \centering
  \includegraphics[width=\linewidth]{pics/31_-2_2}
  \caption{A subfigure}
%  \label{fig:sub1}
\end{subfigure}%
\begin{subfigure}{.5\textwidth}
  \centering
  \includegraphics[width=\linewidth]{pics/31_-2_2}
  \caption{A subfigure}
%  \label{fig:sub2}
\end{subfigure}
\caption{A figure with two subfigures}
%\label{fig:test}
\end{figure}

\newpage

\subsection{La dynamique de chaque robot dépend simplement de l’autre
robot}

\begin{equation}
\centering
\left\{\begin{split}
\dot{w}(t) &= aw(t) \\
\dot{e}(t) &= bw(t) \\
\end{split}\right.
 \end{equation}

\begin{equation}
\centering
A = \left[
\begin{array}{cc}
a & 0\\
b & 0
\end{array}
\right]
 \end{equation}

\begin{figure}
\centering
\begin{subfigure}{.5\textwidth}
  \centering
  \includegraphics[width=\linewidth]{pics/41}
  \caption{A subfigure}
%  \label{fig:sub1}
\end{subfigure}%
\begin{subfigure}{.5\textwidth}
  \centering
  \includegraphics[width=\linewidth]{pics/42}
  \caption{A subfigure}
%  \label{fig:sub2}
\end{subfigure}
%\caption{A figure with two subfigures}
%\label{fig:test}
\end{figure}

\begin{figure}
\centering
  \begin{subfigure}{.5\textwidth}
  \centering
  \includegraphics[width=\linewidth]{pics/41_-2_2}
  \caption{A subfigure}
%  \label{fig:sub1}
\end{subfigure}%
\begin{subfigure}{.5\textwidth}
  \centering
  \includegraphics[width=\linewidth]{pics/41_-2_2}
  \caption{A subfigure}
%  \label{fig:sub2}
\end{subfigure}
\caption{A figure with two subfigures}
%\label{fig:test}
\end{figure}

\newpage

\subsection{Bonus : a n’est pas une constante mais dépend du temps}

On pose a(t) = max(0, 100 − 0.1t).

\begin{equation}
\centering
\left\{\begin{split}
\dot{w}(t) &= a(t)w(t) + be(t)\\
\dot{e}(t) &= bw(t) + a(t)e(t) \\
\end{split}\right.
 \end{equation}

\section{typologie des systèmes}
pour a = -0.15 et b = 0.9
1 noeud instable dégénéré
2 selle
3 centre
4 noeud instable dégénéré

pour a et b = 0.1
1 noeud instable dégénéré
2 ligne équilibre instable
3 noeud stable dégénéré
4 noeud instable dégénéré

pour a et b = -0.1
1 noeud stable dégénéré
2 ligne équilibre stable
3 noeud instable dégénéré
4 noeud stable dégénéré

\newpage
\section{Conclusion}
Nous pouvons conclure des 5 études de cas que les deux robots ont

\end{document}
